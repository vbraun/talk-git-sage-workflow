\subsection{Setting Up}


\begin{frame}[fragile]
  \frametitle{Who Are You?}

  Your name and email address become part of the commit message 
  \begin{itemize}
  \item<2-> Global configuration stored in $\sim$\verb#/.gitconfig#. Either
    open in your favorite editor to add
    \begin{shell}
    [user]
        name = Your Name
        email = you@host.com
    \end{shell}
  \item<3-> or via the command line:
    \begin{shell}
git config --global user.name "Your Name"
git config --global user.email you@host.com
    \end{shell}    
  \end{itemize}

\end{frame}


\begin{frame}[fragile]
  \frametitle{Trac Account}
  
  To contribute to Sage, you need 
  \begin{itemize}
  \item<2-> a trac account, see instructions at \url{http://trac.sagemath.org}
  \item<3-> upload your ssh \emph{public} key to the trac server
  \item<4-> This is described in detail in
    \url{http://sagemath.github.io/git-developer-guide/trac.html#authentication},
    a temporary copy of the new developer guide.
  \end{itemize}
  
\end{frame}



\subsection{Using Git for Sage}


\begin{frame}
  \frametitle{Obtaining the Sage Sources}
  
  \begin{itemize}
  \item<1->
    Download the Sage git repository from github:\\
    \cmd{git clone git://github.com/sagemath/sage.git}
  \item<2->
    Setup the ``trac'' remote:\\
    {
      \cmd{cd sage}\\
      \cmd{git remote add trac}\\
      \hfill
      \cmd{ssh://git@trac.sagemath.org:2222/sage.git -t master}
    }
  \item<3-> Note: the \cmd{-t master} means to only fetch the master
    branch by default 
    \begin{itemize}
    \item<4-> Pro: Avoids downloading all branches on trac; Faster and
      less clutter
    \item<5-> Con: You have to tell git which branches to download
    \end{itemize}
  \end{itemize}
\end{frame}




\begin{frame}
  \frametitle{Downloading a Branch from Trac}

   \begin{alertblock}{Temporary change}
     You should use the \cmd{public/sage-git/master} branch for
     now. When the git transition is finished, it well be just
     \cmd{master}.
   \end{alertblock}
   
   So, first get this branch:
   \begin{itemize}
   \item<1->
     Tell git which branch to download:\\
     \cmd{git fetch trac public/sage-git/master}
   \item<2->
     Create a new local branch from what you just downloaded:\\
     \cmd{git checkout -b trac\_master FETCH\_HEAD}
   \end{itemize}
   \onslide<3->{
     Then build Sage as usual (run \cmd{make})
   }
\end{frame}


\subsection{Integration with Sage Trac}

\begin{frame}
  \frametitle{Uploading Changes}

  \begin{itemize}
  \item<1-> Now edit files and commit changes. Just like with any other git
    repository.
  \item<2-> If you have a (new or existing) ticket, fill in the
    ``Branch:'' field with the name that you will be using to upload.
  \item<3-> The remote branch name must be \cmd{u/user/description}, where
    \begin{itemize}
    \item<3-> \cmd{user} is your trac username
    \item<3-> \cmd{description} is a free-form short description (and can
      include further slashes)
    \end{itemize}
  \item<4-> 
    When you are ready to share, upload to trac:\\
    \cmd{git push --set-upstream trac}\\
    \hspace{2cm}\cmd{my\_branch:u/user/description}
  \item<5->
    Slightly different push command for subsequent uploads:\\
    \cmd{git push trac HEAD:u/user/description}
  \end{itemize}
\end{frame}



\begin{frame}
  \frametitle{Using Trac}

  \begin{itemize}
  \item<1-> When you push to a trac ticket, the ``Commit:'' field on the
    trac ticket is automatically filled out.
  \item<2-> The ``Branch:'' field is color coded:
    \begin{itemize}
    \item \textcolor{green}{Green} means that it applied cleanly to the current master.
    \item \textcolor{red}{Red} means that there is a conflict.
    \end{itemize}
  \item<3-> If you click on the ``(Commits)'' link under/next to the
    branch, you can see the list of commits.
  \item<4-> Download any branch for the first time as on the ``Downloading
    a Branch from Trac'' slide.
  \item<5-> To get changes, use \cmd{git pull trac u/user/description}
  \end{itemize}
\end{frame}


{
  \usebackgroundtemplate{\includegraphics[width=\paperwidth]{images/trac_screenshot}}
  \begin{frame}[plain]
  \end{frame}
}




\begin{frame}[fragile]
  \frametitle{Merging vs.\ Rebasing}

  While you are working on \cmd{my\_branch}, Sage development
  continues.
\begin{verbatim}
             X---Y---Z my_branch
            /
       A---B---C---D master
\end{verbatim}
  \onslide<2->{Two ways to update:}
  \begin{itemize}
  \item<2-> Merge: \cmd{git merge master}
\begin{verbatim}
             X---Y---Z---W my_branch
            /           /
       A---B---C-------D master
\end{verbatim}
  \item<3-> Rebase: \cmd{git rebase master}
\begin{verbatim}
                     X'--Y'--Z' my_branch
                    /
       A---B---C---D master
\end{verbatim}
  \end{itemize}
  
\end{frame}





\begin{frame}[fragile]
  \frametitle{Rebasing}
  
    \begin{itemize}
    \item<1-> Rebase: \cmd{git rebase master}
\begin{verbatim}
                     X'--Y'--Z' my_branch
                    /
       A---B---C---D master
\end{verbatim}
    \item<2-> Pro: Clean history.
    \item<3-> Con: Since the SHA1 hash includes the hash of the parent,
      all commits change.
    \item<4-> Only ever use rebase if nobody else has used one of your
      \texttt{X}, \texttt{Y}, \texttt{Z} commits to base their
      development on.
    \item<5-> Only rebase commits that you have not yet pushed to trac.
  \end{itemize}
  
\end{frame}





\begin{frame}[fragile]
  \frametitle{Merging}
  
    \begin{itemize}
  \item<1-> Merge: \cmd{git merge master}
\begin{verbatim}
             X---Y---Z---W my_branch
            /           /
       A---B---C-------D master
\end{verbatim}
      \item<2-> Pro: None of the existing commits changes
      \item<3-> Con: Introduces a new commit \texttt{W} that will be in the
        \cmd{git log} history forever.
      \item<4-> When you push to trac, the extra commit propagates to your
        collaborators. 
      \item<5-> When in doubt, use merge instead of rebase.
      \item<6-> No new features in \texttt{master} that you depend on and
        no conflicts? Do nothing. Don't create useless merges.
  \end{itemize}
  
\end{frame}



\begin{frame}
  \frametitle{Reviewing Commits}

  \begin{itemize}
  \item<1-> Trac tickets are abstract goals to meet.
  \item<2-> Commits are individual changes of the sources.
  \item<3-> There is only a map \textcolor{blue}{ticket $\to$ subset of
      all commits}, namely all parents of the commit listed on the
    ``Commit:'' trac field.
  \item<4-> In particular, a commit can be part of multiple tickets.
  \end{itemize}

  \onslide<5->{
    \begin{block}{Commits to review}
      The ticket commit and all parent commits leading to the ticket are
      part of the review. Except for commits that are already merged
      into Sage:
      \\
      \cmd{git log <branch-or-sha1> \textasciicircum master}
    \end{block}
  }
\end{frame}


\begin{frame}
  \frametitle{Dependencies and Reviewing Commits}

  \begin{itemize}
  \item<1->
    You can list the history excluding dependencies:\\
    \cmd{git log <branch> \textasciicircum master \textasciicircum
      <dep1> \textasciicircum <dep2>}
  \item<2-> But: When your ticket is merged, all parent commits are
    merged.
  \item<3-> Whether any particular parent is part of a dependency ticket
    can change as the dependency ticket evolves.
  \item<4-> In particular, you might end up with abandoned commits from a
    dependency.
  \item<5-> Hence: All parent commits are part of the review. 
  \item<6-> To simplify review, start with the trac dependencies and have
    them merged into Sage.
  \end{itemize}
\end{frame}




%%% Local Variables:
%%% TeX-master: "talk.tex"
%%% eval: (TeX-PDF-mode 1)
%%% End:

